\centering \section*{Введение}
\begin{flushleft}

Реляционные СУБД перед исполнением SQL-запроса строят \emph{план} — определяют порядок соединения таблиц (\textit{join order}), 
выбирают алгоритмы и точки использования индексов. От качества плана зависит время выполнения и расход ресурсов (CPU, память, 
дисковый и сетевой ввод-вывод). Ключевая подзадача — выбор порядка соединений — комбинаторно взрывоопасна и в общем виде 
NP-трудна: исчерпывающий перебор становится непрактичным уже при умеренном числе таблиц.

Комбинаторная природа задачи видна через простую оценку. Если соединения выполнять бинарно, форма плана задаётся бинарным деревом 
с \(n\) листьями. Таких структур \(C_{n-1}\), где \(C_k\) — число Каталана; асимптотически \(C_n \sim \dfrac{4^n}{n^{3/2}\sqrt{\pi}}\). 
Даже без учёта перестановок самих отношений это уже экспоненциальный рост пространства планов. Дополнительно реальные запросы 
ограничивают допустимые перестановки (по графу соединений и по семантике соединений), что ещё больше усложняет задачу.

Удобная модель — \emph{(гипер)граф соединений}. Вершины — отношения (таблицы), рёбра — предикаты соединений. При предикатах, 
затрагивающих более двух таблиц, естественен гиперграф. Топология здесь существенна: цепочки, «звёзды», плотные циклы или наличие 
гиперрёбер радикально меняют «стоимость» поиска хорошего порядка. Для ациклических гиперграфов известны полиномиальные 
многошаговые процедуры исполнения (в духе алгоритма Яннакакиса), а для общих структур — оценки сложности через параметры, наподобие 
ширины декомпозиции. Тем не менее промышленным оптимизаторам чаще приходится строить бинарные планы и подбирать порядок 
соединений под стоимостную модель.

Ситуацию осложняют внешние соединения (\texttt{OUTER JOIN}): они нарушают полную ассоциативность и коммутативность и жёстко 
ограничивают допустимые переупорядочивания без изменения смысла. В результате оптимизатор должен одновременно учитывать 
(i) топологию (гипер)графа, (ii) семантические ограничения порядка, (iii) неточную стоимостную модель и 
(iv) ограниченный \emph{временной бюджет} на планирование.

На практике индустриальные СУБД комбинируют подходы. Для малого числа таблиц применяют динамическое программирование (перебор 
подмножеств с запоминанием лучших частичных планов). При росте \(n\) переключаются на эвристики и стохастические процедуры: жадные 
стратегии, локальные улучшения, генетические алгоритмы и их простые сочетания. Типичное правило — «по числу таблиц»: до порога 
используется DP, выше — эвристики. У такого правила два недостатка: оно игнорирует форму (гипер)графа и не учитывает бюджет 
времени на поиск, хотя именно эти факторы критичны.

\textbf{Идея работы} — методы \emph{адаптивного планирования}: выбирать и комбинировать алгоритмы \emph{в зависимости от топологии соединений и выделенного бюджета времени} на оптимизацию. Практически это означает:
\begin{enumerate}
  \item ввести быструю метрику «топологической сложности» подзадачи (по графу/гиперграфу соединений и ограничениям \texttt{OUTER JOIN});
  \item на её основе задать \emph{бюджет перебора} (сколько допустимо генераций и оценок планов);
  \item динамически \emph{переключаться} между стратегиями (DP, жадные, стохастические, гибридные), в том числе \emph{на лету} при переходе к подзадачам иной структуры;
  \item гарантировать «безопасное» завершение: при исчерпании бюджета возвращать план предсказуемого качества вместо незавершённого поиска.
\end{enumerate}

\textbf{Научная значимость} состоит в том, что вместо статического «разделения труда по \(n\)» предлагается топология- и бюджет-осознанный контроль над поиском: алгоритм выбора порядка соединений становится процедурой управления вычислительным бюджетом на графе, а не просто выбором одной фиксированной техники. Это позволяет систематизировать уже известные эвристики через единый интерфейс бюджета и сравнивать их «на равных» на классах топологий (цепи, звёзды, циклы, смешанные гипрерёбра).

\textbf{Практическая часть} — реализация прототипа в открытом планировщике (ядро СУБД с открытым кодом) с добавлением: (а) оценки топологической сложности (например, через плотность рёбер/гиперрёбер, наличие «жёстких» компонент из-за \texttt{OUTER JOIN}, ориентировочную ширину декомпозиции); (б) бюджетируемых операторов поиска (ограничение на число вызовов генерации/оценки частичных планов); (в) переключателя стратегий на границах подзадач (цепные фрагменты, почти-звёздные центры, малые циклы и т.\,п.).

\textbf{Оценка} — на бенчмарках TPC-H и TPC-DS: сравнение времени планирования, качества планов (время выполнения, пиковая память, объём I/O) и устойчивости результатов по классам топологий. Ожидаемый эффект — более стабильное качество планов при контролируемом времени оптимизации, особенно на «неудобных» структурах графа и при наличии внешних соединений.

Таким образом, работа соединяет математически понятную комбинаторику (числа Каталана, ацикличность/циклы, гиперрёбра и ограничения на перестановки) с инженерной задачей управления вычислительным бюджетом в оптимизаторе запросов. Результат — набор адаптивных эвристик и правил переключения, воспроизводимая реализация и экспериментальные данные, пригодные для дальнейшего развития в промышленных СУБД.

\end{flushleft}