
\centering \section*{Аннотация}
\begin{flushleft}
Работа посвящена разработке новых эвристических методов планирования запросов в реляционных СУБД. В литературе известно достаточно много алгоритмов планирования, 
но в индустриальных применениях известны лишь базовые подходы, опирающиеся на динамическое программирование, жадные и генетические алгоритмы и их простейшие 
эвристические комбинации, позволяющие выбирать тот или иной алгоритм в зависимости от числа таблиц в запросе. Данная работа преследует цель получить глубокое понимание 
того, как разные алгоритмы возможно комбинировать в ходе планирования запроса, делая переключение между алгоритмами в зависимости от топологии соединений и выделенного 
временного бюджета на планирование. Ожидаемым результатом работы является решение оптимизационной задачи планирования с помощью новых эвристик, реализация решения в 
виде модификации планировщика в ядре реляционной СУБД с открытым кодом и экспериментальная оценка на известных индустриальных бенчмарках.
\end{flushleft}